\documentclass{article}
\usepackage[utf8]{inputenc}
\usepackage{geometry}
\usepackage{hyperref}

\title{Projekt aplikacji pokazującej opóźnienia autobusów}
\author{Adam Kliś}
\date{\today}

\begin{document}

\maketitle

\section{Wprowadzenie}
Niniejszy dokument stanowi podsumowanie projektu aplikacji mającej na celu pobieranie danych z API ZTM Warszawskiego, przetwarzanie tych danych, wyświetlanie mapy przystanków oraz obliczanie średniego opóźnienia autobusów na wybranej trasie. Projekt zakładał również przechowywanie zebranych danych w bazie przez długi okres czasu, przekraczający miesiąc.

\section{Cel Projektu}
Głównym celem projektu było opracowanie aplikacji umożliwiającej użytkownikom:
\begin{itemize}
    \item Pobieranie danych z API ZTM Warszawskiego, w tym informacji o przystankach, rozkładach jazdy i opóźnieniach.
    \item Wybór dwóch przystanków, linii autobusowej, kierunku oraz okresu czasu.
    \item Wyświetlenie mapy z zaznaczonymi przystankami.
    \item Obliczenie średniego opóźnienia autobusów na wybranej trasie w określonym czasie.
    \item Przechowywanie zebranych danych w bazie danych przez długi okres (powyżej miesiąca).
\end{itemize}

\section{Implementacja}
Aplikacja została zaimplementowana z wykorzystaniem dwóch języków programowania:
\begin{itemize}
    \item Do pobierania danych z API ZTM Warszawskiego i zapisu do bazy danych użyto języka C#.
    \item Aplikacja interfejsu użytkownika została napisana w języku Python.
\end{itemize}
W części C# wykorzystano bibliotekę HttpClient do komunikacji z API oraz Entity Framework do obsługi bazy danych.
Baza danych będzie zawierała tabele z rozkładami, trasami, godzinami poprawnymi i rzeczywistymi przyjazdu autobusów na każdy przystanek. 
Usługi po stronie serwera będą w napisane w C#, a sama aplikacja okienkowa będzie w pythonie.
Do bazy danych trafiają tylko czasy, kiedy autobus zatrzyma się na przystanku.
Aktualizacja rozkładów będzie robiona codziennie w nocy. 


\section{Pytania}
\begin{enumerate}
    \item Czy można by było dostać komletne polecenie do projektu?
    \item Czy może być mniej testerów niż 10?
    \item Jak mają być uwzględniane przystanki na żądanie?
    \item W jaki sposób mamy postawić serwer? Czy musi być to komputer zewnętrzny, czy dostaniemy dostęp do komputera uczelnianego?
    \item Co jeżeli autobus nie przejedzie trasy?
    \item Jak mamy ustalić, o której godzinie konkretny autobus ma być na konkretnym przystanku, skoro możemy pobrać tylko listę wszystkich czasów o których autobus danej linii będzie na przystanku?
    \item Jak mamy rozpoznać autobusy z bardzo dużym opóźnieniem (30 minut) i rozróżnić je od następnych autbusów tej samej linii?
    \item Jak mamy rozwiązać problem z autobusami, które jadąc w jednym kierunku przejeżdżają przez tą samą drogę w obydwu kierunkach?
    \item Jak mamy ustawić trasę dla autobusów, które mają inne trasy w zależności od kierunku
\end{enumerate}

\end{document}
